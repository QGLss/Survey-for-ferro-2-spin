\documentclass[UTF8]{ctexart}
\usepackage{fancyhdr}
\usepackage{extramarks}
\usepackage{amsmath}
\usepackage{amsthm}
\usepackage{amsfonts}
\usepackage{tikz}
\usepackage[plain]{algorithm}
\usepackage{algpseudocode}
\usepackage{color}
\topmargin=-0.45in
\evensidemargin=0in
\oddsidemargin=0in
\textwidth=6.5in
\textheight=9.0in
\headsep=0.25in

\linespread{1.1}

\DeclareMathOperator{\prob}{Pr}
\DeclareMathOperator{\expr}{\mathbf{E}}
\DeclareMathOperator{\indi}{\mathbf{I}}
\DeclareMathOperator{\var}{\mathbf{Var}}


\newtheorem{defn}{Definition}
\newtheorem{theorem}{Theorem}

\begin{document}
\section{摘要}

\subsection{铁磁性双自旋系统配分函数近似算法研究现状}
     铁磁性双自旋系统配分函数的计算是统计物理,理论计算机,应用数学等领域的重要问题。本文目的在于完整地呈现
    近似计算铁磁性双自旋系统配分函数的相关技术现状,以及指出该课题当前的一些开放性问题。本文首先介绍铁磁性
    双自旋系统的配分函数的定义及其相关的采样问题。然后我们提出对该问题计算相变点的讨论,最后是对各种近似计算的
    技术的讨论。

\section{双自旋系统配分函数及其采样问题}
    

\section{近似算法技术}

 $\indent$ It is well known that exact computation for the partition function of the spin system is $\#$P-hard\cite{phard},
except for the very restricted settings.  Thus, we are 
interested in the approximability of computing the partition function for spin system. We mainly focus on fully polynomial-time approximation scheme(FPTAS).
\begin{defn}
    \textbf{(FPTAS)} For any given parameter $\epsilon>0$, the algorithm outputs a number $\hat Z$ such that $Z_G \cdot e^{-\epsilon}\leq \hat Z \leq Z_G\cdot e^{\epsilon}$
    and runs in time $poly(n,\frac{1}{\epsilon})$, where n is the size of G.
\end{defn}

The randomized relaxation of FPTAS is called fully polynomial-time randomized approximation scheme(FPRAS) defined as follows,

\begin{defn}
    \textbf{(FPRAS)} For any given parameter $\epsilon>0$, the algorithm outputs a number $\hat Z$ such that $Pr(Z_G \cdot e^{-\epsilon}\leq \hat Z \leq Z_G\cdot e^{\epsilon})\geq \frac{3}{4}$
    and runs in time $poly(n,\frac{1}{\epsilon})$, where n is the size of G.
\end{defn}

\section{Counting and the Marginal probablity (q=2)}

 
\bibliographystyle{plain}
\bibliography{survey}
\end{document}